\documentclass[a4paper,10pt]{article}
% UTF8 Charset für Umlaute/Sonderzeichen
\usepackage[utf8]{inputenc}
% Deutsche Namen und Datumsformat
\usepackage[ngerman]{babel}
%Quellenangaben
\usepackage{hyperref}
%Fancy Mathesymbole
\usepackage{amssymb}
\usepackage{dsfont}
%Aufzählung
\usepackage{paralist}
%Pfeile
\usepackage{amsmath}
\makeatletter
\newcommand{\xRightarrow}[2][]{\ext@arrow 0359\Rightarrowfill@{#1}{#2}}
\makeatother
%Krams
\usepackage{amsmath}
\usepackage{amsthm}
%Gewitter
\usepackage{ stmaryrd }
% Algorithmen Pseudocode
\usepackage[]{algorithm2e}
% Bilder
\usepackage{graphics}
\usepackage{wrapfig}
%GeoGebra
\usepackage{tikz}
\usepackage{tkz-euclide}
\usetikzlibrary{arrows,automata,positioning}

\tikzset{
	state/.style={
		circle,
		draw=black, very thick,
		minimum height=2em,
		inner sep=2pt,
		text centered,
	},
}



\newcommand{\bigo}{\ensuremath{\mathcal{O}}}
\newcommand{\BT}{\ensuremath{\Theta}}
\newcommand{\N}{\ensuremath{\mathbb{N}}}
\newcommand{\Set}{\ensuremath{\mathcal{S}}}
\newcommand{\PMenge}{\ensuremath{\mathcal{P}}}
\newcommand{\Point}[2]{\ensuremath{\begin{pmatrix}#1\\#2\end{pmatrix}}}
\newcommand{\xp}{\ensuremath{x_{post}}}
\newcommand{\yp}{\ensuremath{y_{post}}}
\newcommand{\zp}{\ensuremath{z_{post}}}


% Hier die Nummer des Blatts, Autoren und Name des Betreuers angeben.
\newcommand{\blatt}{8}
\newcommand{\autor}{Edenfeld, Lemke, Moser, Schinke}
\newcommand{\betreuer}{Carina Pilch}
\usepackage{../abgabe}

\begin{document}
	% Seitenkopf mit Informationen
	\kopf
	
	\aufgabe 1


\begin{itemize}
	\item[a)] $AG(rec \rightarrow (A(\neg wait)U^{\leq2} ack))$ 
	\item[b)] $EF^{<\infty} rec$
	\item[c)] $E(wait \rightarrow EF^{\leq 10}(dist \wedge x=0))$	
\end{itemize}

Die TCTL Formel a) wird nicht vom Automaten erfüllt, da es einen Pfad gibt, in dem 2 Minuten im Zustand \textit{rec} und zusätzlich eine weitere Minute in \textit{fwd} gewartet wird, wodurch \textit{ack} erst nach insgesamt 3 Minuten erreicht werden würde. \\

Die TCTL Formel b) wird nicht vom Automaten erfüllt, da es möglich ist immer nach 10 Minuten in den Zustand $dist$ zu wechseln, die bei Minute 11 einkommende Nachricht zu verpassen und wieder zurück nach $wait$ zu springen, wodurch effektiv keine Nachricht durchkommt. \\

Die TCTL Formel c) wird vom Automaten erfüllt, da es möglich ist bei Minute 10 in den Zustand $dist$ zu wechseln und dadurch zeitgleich eine eingehende Nachricht zu verpassen.

	
	\aufgabe 2



	
	\aufgabe 3



	 
\end{document}

