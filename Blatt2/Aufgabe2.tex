\aufgabe 2

\paragraph{a)}\mbox{} \\
\begin{tikzpicture}[->,>=stealth',shorten >=1pt,auto,node distance=2.8cm,
semithick]
\node[state,minimum size=2.2cm] (l01) 
{\begin{tabular}{c}
	$l_0$\\
	$x=0$				
	\end{tabular}};

\node[state,minimum size=2.2cm] (l02) [right of=l01]
{\begin{tabular}{c}
	$l_0$\\
	$0<x<1$				
	\end{tabular}};

\node[state,minimum size=2.2cm] (l03) [right of=l02]
{\begin{tabular}{c}
	$l_0$\\
	$x=1$				
	\end{tabular}};

\node[state,minimum size=2.2cm] (l04) [right of=l03]
{\begin{tabular}{c}
	$l_0$\\
	$x>1$				
	\end{tabular}};



\path (l01) edge        node {$\tau$} (l02)
(l02) edge              node {$\tau$} (l03)
(l03) edge              node {$\tau$} (l04)
(l01) edge [loop above] node {$\alpha$} (l01)
(l02) edge [loop above] node {$\alpha$} (l02)
(l03) edge [loop above] node {$\alpha$} (l03);


\end{tikzpicture}

\paragraph{b)}\mbox{} \\

Es existiert ein Zenopfad in $\tau_1$, da der $\alpha$-Übergang unendlich mal durchgeführt werden kann ohne Zeit verstreichen zu lassen. Somit könnten die Zustände $(l_0, x=0), (l_0, 0<x<1), (l_0, x=1)$ jeweils unendlich viele $\alpha$-Übergänge durchführen. \\

Um dieses Zenoverhalten nicht zu ermöglichen, wird folgende $TCTL$-Formel benötigt: 